\documentclass[english,red]{beamer}
%\documentclass[russian,english,hyperref={bookmarks=false},red]{beamer}
\mode<presentation> {
  \usetheme{Warsaw}
  %\usecolortheme{beetle}
  %\usecolortheme{lily}
  % or ...

  %\setbeamercovered{transparent}
  % or whatever (possibly just delete it)
}

\usepackage[T2A]{fontenc}
\usepackage[cp1251]{inputenc}
\usepackage[russian,english]{babel}
\usepackage{graphicx}
%\usepackage{moreverb}
\usepackage{amsfonts,amssymb}
%\usepackage{concrete}
\usepackage{cancel}


%\renewcommand{\partname}{�����}
%\newtheorem{defin}{Definition}
%\newtheorem{thm}{Theorem}
%\newtheorem{cor}{Corolarry}
%\newtheorem{prb}{Problem}

\newcommand{\ignore}[1]{}

%\def\pyth{\texttt{Python}}
%\newcommand{\bm}[1]{\boldsymbol{#1}}
%\newcommand{\pder}[2]{\frac{\partial #1}{\partial #2}}
%\newcommand{\scal}[2]{\left\langle{#1}, {#2}\right\rangle}
%\newcommand{\scalv}[2]{\langle{\vec{#1}}, {\vec{#2}}\rangle}
%\newcommand{\round}[1]{\left\lfloor\left. {#1} \right\rceil\right.}
%\def\fit{\mathrm{Fitness}}
%\def\cons{\mathrm{Cons}}
%\def\amax{\mathrm{argmax}}
%\def\amin{\mathrm{argmin}}
%\def\MAP{\mathrm{MAP}}
%\def\pr{\mathrm{Pr}}
%\def\pa{\mathsf{pa}}
%\def\ch{\mathsf{ch}}
%\def\fam{\mathsf{Fam}}
%\def\clique{\mathsf{Clique}}
%\def\SS{\scr S}
%\def\s{{\bf s}}
%\def\E{{\bf E}}
%\def\M{{\cal M}}
%\def\B{{\cal B}}
%\def\b{{\bf b}}
%\newcommand{\sn}[1]{\s_{-#1}}
%\newcommand{\Sn}[1]{{\bf \Sigma}_{-#1}}
\def\X{{\cal X}}
\def\T{{\cal T}}
\def\F{{\cal F}}
\def\H{{\cal H}}
\def\G{{\cal G}}
\def\A{{\cal A}}
\def\P{{\cal P}}
\def\C{{\cal C}}
\def\D{{\cal D}}
\newcommand{\Xn}[1]{\X_{-#1}}
\def\bx{{\bf x}}
\def\bp{{\bf p}}
\def\bX{{\bf X}}
\def\bW{{\bf W}}
\def\by{{\bf y}}
\def\be{{\bf e}}
\def\bg{{\bf g}}
\def\bY{{\bf Y}}
\def\sgn{\mathrm{sgn}}
\newcommand{\bxn}[1]{{\bf x}_{-#1}}
\newcommand{\bXn}[1]{{\bf X}_{-#1}}
\renewcommand{\vec}[1]{{\bf #1}}
\def\calth{{\cal\theta}}
\newcommand{\calthn}[1]{\calth_{-#1}}
\def\calTh{{\cal\Theta}}
\newcommand{\calThn}[1]{\calTh_{-#1}}
\def\O{{\cal O}}
\def\cost{\mathrm{cost}}
\def\P{{\cal P}}
\def\R{{\cal R}}
\def\nei{{\mathrm{nei}}}
\def\Q{{\bf Q}}
\def\bM{{\bf M}}
\def\n{\bar}
\def\ti{\tilde }
%\newcommand{\der}[2]{\frac{d #1}{d #2}}
%\newcommand{\pdersec}[2]{\frac{\partial^2 #1}{\partial {#2}^2}}
%\newcommand{\pderinline}[2]{{\partial #1}/{\partial #2}}
%\newcommand{\ptc}[2]{p(\tilde #1|\tilde #2)}
%\newcommand{\pc}[2]{p(#1|#2)}

%\newcommand{\ket}[1]{\left|\left. #1 \right\rangle\right.}

\author[Nurk, Bankevich]{Sergey Nurk, Anton Bankevich}
\date{ABLab, December 15, $2011$}

%\AtBeginSection[] {
%  \begin{frame}<beamer>
%    \frametitle{Outline}
%    \tableofcontents[currentsection]
%  \end{frame}
%}

%\AtBeginPart{
%\frame{\partpage}
%  \begin{frame}
%    \frametitle{����������}
%    \tableofcontents[part=\thepart]
%  \end{frame}
%}

%\newlength{\picheight}
%\picheight = \textheight \addtolength{\picheight}{-2cm}

%\newlength{\twocolwidth}
%\twocolwidth = .5\textwidth

\newcommand{\jpg}[2]{\begin{center}\includegraphics[height=#2]{jpg/#1.jpg}\end{center}}
\newcommand{\bigjpg}[1]{\begin{center}\includegraphics[height=\picheight]{jpg/#1.jpg}\end{center}}
\newcommand{\widjpg}[1]{\begin{center}\includegraphics[width=\textwidth]{jpg/#1.jpg}\end{center}}
\newcommand{\twocoljpg}[1]{\begin{center}\includegraphics[width=\twocolwidth]{jpg/#1.jpg}\end{center}}
\newcommand{\twocol}[2]{\begin{columns} \column{\twocolwidth} #1 \column{\twocolwidth} #2 \end{columns}}
\newcommand{\twocolwjpg}[2]{\begin{columns} \column{\twocolwidth} \twocoljpg{#1} \column{\twocolwidth} #2\end{columns}}

%\renewcommand{\verbatimtabsize}{4}
%\newcommand{\src}[1]{{\footnotesize\verbatimtabinput{#1}}}

%\setbeamercolor{assignment}{fg=black, bg=red!50}
%\newcommand{\hometask}[1]{\begin{beamercolorbox}{assignment}
%\vspace{.1cm}

%{\bf Exercise.}
%#1 \vspace{.1cm}
%\end{beamercolorbox}}

\newcommand{\fr}[2]{\begin{frame}\frametitle{#1} #2 \end{frame}}
\newcommand{\frt}[2]{\begin{frame}[t]\frametitle{#1} #2 \end{frame}}
\newcommand{\fri}[2]{\begin{frame}\frametitle{#1}\begin{itemize} #2 \end{itemize}\end{frame}}
\newcommand{\frit}[2]{\begin{frame}[t]\frametitle{#1}\begin{itemize} #2 \end{itemize}\end{frame}}
\newcommand{\frih}[3]{\begin{frame}\frametitle{#1}\begin{itemize} #2 \end{itemize} \hometask{ #3} \end{frame}}

%\newcommand{\finalslide}[1]{\begin{frame}{ \begin{center}\bf\Large\color{red} Thank you for your attention!\end{center} #1 }

% \begin{itemize}
% \item Lecture notes � ������ ����� ���������� �� ����
% homepage: \\ {\footnotesize
% \texttt{http://logic.pdmi.ras.ru/$\sim$sergey/}}
% \item ���������� ����� ���������,
% ������� ����������, ����� ��������� ������� � ������ �� �������: \\
% \texttt{sergey@logic.pdmi.ras.ru}, \texttt{snikolenko@gmail.com}
% \item �������� � �� \includegraphics[height=.3cm]{userinfo.jpg}{\color{red}\bf smartnik}.
% \end{itemize}
%\end{frame}}

\title{SPAdes overview}
\begin{document}
\selectlanguage{english}
\begin{frame}
\titlepage
\end{frame}


\fri{SPAdes project} {
\item Currently largest project of ABLab. 11 people.
\item Started 10 months ago, intensively developed since then.
\item Incorporates new both algorithmic and design approaches.
\item Currently for small (up to 50Mb) genomes only. Working on scaling.
\item 3 related papers submitted to RECOMB.
\item Sources will be available soon.
%MOVE!!!\item Everything is currently designed to show good results on single-cell data
}

\fr{Motivation} {
Lots of assemblers, but:
\begin{itemize}
\item some badly supported or even dead %and don't move forward
\item some have poor software design for experimenting with new algorithms or extending to new type of data
\item no universal assembler for multiple libraies, long counter-intuitive protocols used instead
\item nobody works well with single-cell data
\end{itemize}
}

\fr{Single-cell data} {
MDA amplification leads to:
\begin{itemize}
\item Higher error rates
\item Highly non-uniform coverage
\item High rates of chimeric reads and chimeric pairs
%\item Often unpleasant insert size distribution (esp. for ``jumping'' libraries)
\end{itemize}

\pause
All these problems need special treatment% (instead of popular ``normalization'' strategies)
}

\fri{SPAdes project units} {
\item Error correction (BayesHammer).
\item Easy to use and extend graph framework.
\item ``Earmark'' approach for space requirments reduction.
\item All flavours of graph processing algorithms.
\item Novel approaches for more precise usage of paired information.
\item Several strategies for repeat resolution (graph untangling). Support for multiple libraries and "jumping libraries" (in progress).
\item Scaffolding (in progress).
\item Assembly quality evaluation.
\item Everything designed to work on single-cell data as well 
}

\fri{De Bruijn graphs... one more time} {
%\item SPAdes utilizes standard de Bruijn graph framework
\item Vertices: k-mers
\item Edges: sequences of k+1-mers
\item Everything becomes more interesting with ``earmarks''
\item Genome corresponds to certain path in de Bruijn graph
\item Resulting contigs: sequences written on edges after graph simplified and repeats resolved
}

\fri{What SPAdes does with single reads} {
\item Error correction
\item Condensed de Bruijn graph construction
\item Graph simplification (remaining ``erroneous'' segments cleaning)
}

\fri{Our strategy to crack single-cell data} {
\item BayesHammer algorithm for error correction
\item Iterative $K-$varying procedure to ``save'' the low covered connections during construction
%\item Graph simplification
}

\fr{What about graph simplification?}{

Most assemblers determine erroneous connections based on their ``coverage''

One has to deal with ``real'' segments of coverage 3 and ``erroneous'' segments of coverage 500

Our kung-fu:
\begin{itemize}
\item Minimal usage of coverage
\item Involved ``erroneous'' edges cleaning algorithms
\item Criterias based on topology of locality
\item Book-keeping during cleaning that gives additional opportunities and allows to ``corrupt'' even more than needed to simplify the future steps
%\item Bookkeeping during bulge removal that allows to ``corrupt'' all the bulges
\end{itemize}
}

%\fri{Bulge removal and bookkeeping} {
%\item Buldges are glued to real path instead of being removed
%\item Information is transfered to real path in such a way that reads that mapped to bulge before would map to real path
%}

%\fri{Chimeric edges removal} {
%\item Most assemblers determine erroneous connections based on coverage
%\item This strategy does not work on single cell
%\item Alternatively erroneous connections can be found based on graph structure under assumption that very long edges are unique
%}

%\fri{Distance estimation} {
%\item Information about distances between edges can not be found precisely just with information from paired reads
%\item Additionally graph structure is used to find the real distances
%\item Lengths of all paths are collected and information form reads chooses one (or several) of them.
%}

\fr{Strategies to deal with} {
Paired reads...

\pause
Mate reads (``Jumping'' libraries)...
%Long single reads?

\pause
Multiple libraries...

\pause
\vspace{2cm} Not covered in this talk!
}

\fr{} {
Thanks for your attention.
}

\end{document}

