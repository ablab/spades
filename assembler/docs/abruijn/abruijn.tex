\documentclass[12pt]{article}

\usepackage{amssymb, amsmath, amsthm}
%\usepackage{fullpage}
\usepackage[final,colorlinks,hyperindex,unicode=true]{hyperref}
\usepackage{tikz}

\begin{document}
\title{A Bruijn Graph Approach}

Choose a hash function $h$ defined on $k$-mers. 
Preferably, make it stable with respect to reverse-complementary 
strings, that is $h(s) = h(s^{RC})$. For example, 
$h(s) = h_1(s) \oplus h_1(s^{RC})$, or $h(s) = \min\{h_1(s), h_1(s^{RC})\}$.


%\end{document}

Now, in each read from the input data, find (about) two $k$-mers 
with the smallest value of $h$.

This read now corresponds to an edge from $A$ to $B$. This edge 
contains the information about its length and (possibly) its content.

Consider the graph of all such vertices and edges.

A contig corresponds to a path in this graph.

1. $h(s)$ += frequency of $s$ in the reads;

Motivation: consider a $k$-mer that is present in many 
places (e.g. part of a repeat). Such $k$-mer is not a vertex 
that's very pleasant to work with. If a read contains some more 
unique $k$-mers, let's rather use them -- this might simplify 
the resulting graph structure.

1a. On the other hand a $k$-mer with an exceptionally low frequency 
has high chances to be just erroneous. A penalty should be put upon 
such $k$-mer as well.

Pros

    * Error-tolerance. If an error doesn't fall into the 
    k-mers with small hash function values, it doesn't bring any wrong info.
    * Mate pairs seem to fit naturally into this concept. 

Contras

    * Does this graph really represent the repeat structure of the genome? 

To do

    * Code this
    * Run against a reference genome and sample read data. Research.
    * N. B. Use GraphViz to explore the resulting graph. 



\end{document}